\documentclass[review]{elsarticle}

\usepackage{lineno,hyperref}
\modulolinenumbers[5]

\journal{Journal of \LaTeX\ Templates}

\bibliographystyle{elsarticle-num}
%%%%%%%%%%%%%%%%%%%%%%%

\begin{document}

\begin{frontmatter}

\title{Summary: Point matching methods for Repetitive Structure in Visual SLAM application}

\begin{abstract}
% This template helps you to create a properly formatted \LaTeX\ manuscript.
\end{abstract}

\begin{keyword}
% \texttt{elsarticle.cls}\sep \LaTeX\sep Elsevier \sep template
% \MSC[2010] 00-01\sep  99-00
\end{keyword}

\end{frontmatter}

\linenumbers

\section{Introduction: Point matching in Visual SLAM}

\paragraph{Data Association and Point Matching in Semantic SLAM} 
Modern systems rely on maximum a posteriori (MAP) estimation which, in the case of visual sensors, corresponds to bundle adjustment (BA), either geometric BA that minimizes feature reprojection error, in feature-based methods, or photometric BA that minimizes the photometric error of a set of selected pixels, in direct methods. 

With the recent emergence of VO systems that integrate loop closing techniques, the frontier between VO and SLAM is more diffuse. The goal of visual SLAM is to use the sensors on-board a mobile agent to build a map of the environment and compute in real time the pose of the agent in that map. In contrast, VO systems put their focus on computing the agent’s ego-motion and not on building a map. The big advantage of a SLAM map is that it allows matching and using in BA previous observations performing three types of data association.

1) Short-term data association: matching map elements obtained during the last few seconds. This is the only data association type used by most VO systems, which forget environment elements once they get out of view, resulting in continuous estimation drift even when the system moves in the same area.

2) Mid-term data association: matching map elements that are close to the camera whose accumulated drift is still small. These can be matched and used in BA in the same way than short-term observations and allow to reach zero drift when the systems move in mapped areas. They are the key to the better accuracy obtained by our system compared against VO systems with loop detection.

3) Long-term data association: matching observations with elements in previously visited areas using a place recognition technique, regardless of the accumulated drift (loop detection), the current area being previously mapped in a disconnected map (map merging), or the tracking being lost (relocalization). Long-term matching allows to reset the drift and to correct the map using pose-graph (PG) optimization or, more accurately, using BA. This is the key to SLAM accuracy in medium and large loopy environments


* Semantically-labeled landmarks address two critical issues of geometric SLAM: data association (matching sensor observations to map landmarks) and loop closure (recognizing previously-visited locations).

Point level.
Semantic points = Descriptor
In theory: Data Association and Optimization
In practical: There are three scale level for point matching

\begin{itemize}
	\item Image level (low)
	\item Local map level (mid)
	\item Global map level (high - for loop closing)
\end{itemize}

Point matching cases:
\begin{itemize}
	\item Odometry: image level
	\item Local bundle adjustment (Local BA).
	\item Relocalization case: Kid-napped robot
\end{itemize}

\paragraph{Challenge in UAV inspection} 
Repetitive structures ...
Weak GPS signals ...
Solution: 
\begin{enumerate}[(i)]
	\item More addition sensors - more constraints
	\item Sophisticating data association method
\end{enumerate}

\paragraph{Outlines} 
In this paper, we ...

\section{Problems solving}

\subsection{Approaches}
- Features descriptor

- ...

- ...

- ...

\subsection{Experiment}


% Here are two sample references: \cite{Feynman1963118,Dirac1953888}.

\section*{References}

% \bibliography{mybibfile}

\end{document}