\documentclass[review]{elsarticle}

\usepackage{lineno,hyperref}
\modulolinenumbers[5]

\journal{Journal of \LaTeX\ Templates}

\bibliographystyle{elsarticle-num}
%%%%%%%%%%%%%%%%%%%%%%%

\begin{document}

\begin{frontmatter}

\title{Summary: Point matching methods for Repetitive Structure in Visual SLAM application}

%\tnotetext[mytitlenote]{Fully documented templates are available in the elsarticle package on \href{http://www.ctan.org/tex-archive/macros/latex/contrib/elsarticle}{CTAN}.}

%% Group authors per affiliation:
%\author{Elsevier\fnref{myfootnote}}
%\address{Radarweg 29, Amsterdam}
%\fntext[myfootnote]{Since 1880.}

%% or include affiliations in footnotes:
% \author[mymainaddress,mysecondaryaddress]{Elsevier Inc}
% \ead[url]{www.elsevier.com}

%\author[mysecondaryaddress]{Global Customer %Service\corref{mycorrespondingauthor}}
%\cortext[mycorrespondingauthor]{Corresponding author}
%\ead{support@elsevier.com}

%\address[mymainaddress]{1600 John F Kennedy Boulevard, Philadelphia}
%\address[mysecondaryaddress]{360 Park Avenue South, New York}

\begin{abstract}
% This template helps you to create a properly formatted \LaTeX\ manuscript.
\end{abstract}

\begin{keyword}
% \texttt{elsarticle.cls}\sep \LaTeX\sep Elsevier \sep template
% \MSC[2010] 00-01\sep  99-00
\end{keyword}

\end{frontmatter}

\linenumbers

\section{Introduction: Point matching in Visual SLAM}

\paragraph{Data Association and Point Matching in Semantic SLAM} 
Semantically-labeled landmarks address two critical issues of geometric SLAM: data association (matching sensor observations to map landmarks) and loop closure (recognizing previously-visited locations).

Point level.
Semantic points = Descriptor
In theory: Data Association and Optimization
In practical: There are three scale level for point matching

\begin{itemize}
	\item Image level (low)
	\item Local map level (mid)
	\item Global map level (high - for loop closing)
\end{itemize}

Point matching cases:
\begin{itemize}
	\item Odometry: image level
	\item Local bundle adjustment (Local BA).
	\item Relocalization case: Kid-napped robot
\end{itemize}

\paragraph{Challenge in UAV inspection} 
Repetitive structures ...
Weak GPS signals ...
Solution: 
\begin{enumerate}[(i)]
	\item More addition sensors - more constraints
	\item Sophisticating data association method
\end{enumerate}

\paragraph{Outlines} 
In this paper, we ...

\section{Problems solving}

\subsection{Approaches}
- Features descriptor

- ...

- ...

- ...

\subsection{Experiment}


% Here are two sample references: \cite{Feynman1963118,Dirac1953888}.

\section*{References}

% \bibliography{mybibfile}

\end{document}